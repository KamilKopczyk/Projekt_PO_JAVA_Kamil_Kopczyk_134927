\chapter{Opis założeń projektu}
\label{chap:opis}


\section{Cele projektu}
Celem projektu było stworzenie systemu do rezerwacji usług łowiska wędkarskiego, plan zakładał budowę aplikacji która zapewni prostą obsługę rezerwacji łowisk, wędek i domków po przez proste schludne GUI jak i przeglądanie historii rezerwacji. Całość ma ułatwić i w pewnym sensie zautomatyzować
obsługę kompleksu łowisk na tyle że klient tylko przyjeżdża odbiera swoją rezerwację i idzie się cieszyć wędkowaniem bez żadnych zbędnych dokumentów do wypełnienia.

\section{Wymagania funkcjonalne}
Aplikacja będzie oferować logowanie się na istniejące już konto w bazie danych MySQL lub rejestracje nowego konta, pozwala również na przeglądanie bez logowania jednak tylko użytkownicy zalogowani mogą korzystać w pełni z aplikacji. Każdy zalogowany użytkownik będzie miał swoją historie rezerwacji która przekazuje mu informacje od kiedy do kiedy trwała dana rezerwacja ile kosztowała i czego dotyczyła. Aplikacja będzie zrobiona bardzo intuicyjnie co pozwoli nawet tym mniej obeznanym z technologią ludziom przejść przez nią bez żadnego problemu z uśmiechem na twarzy.


\section{Wymagania niefunkcjonalne}
Poza samą funkcjonalnością założeniem jest żeby z aplikacji korzystało się efektywnie, przyjemnie i bezproblemowo. Dlatego zarówno osoba nie zalogowana jak i zalogowana ma czuć że program działa intuicyjnie i płynie bez żadnego zbędnego czekania na odpowiedź aplikacji. Ważne jest też żeby aplikacja była przygotowana na nieprzewidziane awarie, takie jak na przykład problem z dostępem do bazy danych. Zapewnione będą również scenariusze awaryjne na takie sytuację jak na przykład nie pobranie jakiegoś id z bazy danych jak i scenariusze na wypadek błędu ze strony użytkownika aplikacji jak na przykład nie prawidłowy format daty lub wybranie w polu startu rezerwacji datę późniejszą niż data rozpoczęcia. Dlatego też system powinien w zrozumiały dla użytkownika sposób po-informować go o zaistniałym problemie. W dodatku dzięki zastosowaniu technologii Java, aplikacja będzie uniwersalna.